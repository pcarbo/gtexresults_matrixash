\section*{Introduction}
Variation in gene expression is an important mechanism underlying susceptibility to complex disease. The simultaneous genome-wide assay of gene expression and genetic variation allows the mapping of the genetic factors that underpin individual differences in quantitative levels of expression (expression QTLs; eQTLs). 
The availability of this information provides immediate insight into a biological basis for disease associations identified through genome-wide association (GWA) studies, and can help to identify networks of genes 
involved in disease pathogenesis (\cite{nicolae_trait-associated_2010, veyrieras_high-resolution_2008}).
Available methods are limited not only in their ability to {\it jointly analyze data on all tissues} to maximize power, but also in simultaneously {\it allowing for both qualitative and quantitative differences among eQTLs} present in each tissue.

Initial approaches to quantify the effect of a particular SNP on gene expression considered only one tissue at a time, and ignored the effect of the SNP on gene expression in other tissues.
This failed to exploit the power of shared genetic variation in effects on expression - i.e. the information that the effect of the gene-snp pair in one tissue can provide about the effect in another- and limited our understanding of multiple-tissue phenotypes.
Furthermore, even past attempts at quantifying heterogeneity of eQTLs using the data across tissues jointly were limited in both the number of tissues considered, and also the level of heterogeneity considered. Qualitative heterogeneity refers to calling a snp `active' or `inactive' in a given tissue. For example, previous attempts at joint analyses referred to the setting in which the gene-snp pair is active in all tissues as `shared' and active in only one as `tissue-specific'  (\cite{flutre_statistical_2013,wen_bayesian_2014}).
However, a QTL may be `active' in all or many tissues and with varying magnitude or sign; we refer to this as quantitative heterogeneity.

Indeed, our initial motivation came from our group's past analysis of GTEx pilot data \cite{consortium_genotype-tissue_2015}, in which we saw evidence that many $(50\%)$ QTLs are shared across all nine tissues. In this context, a QTL was called based on whether it demonstrates significant posterior probability of being active in a particular tissue. Applying our previous hierarchical model (`eQTL-BMA') to the dataset from Dimas {\it et al} \cite{dimas_common_2009} with 3 tissues, we found just 8\% of eQTLs were specific to a single tissue, with an estimated 88\% of eQTLs being common to fibroblasts, LCL cells and T-Cells \cite{flutre_statistical_2013}. Not all eQTLs are shared by all tissues; some tissues may share eQTLs more than others. To allow for this, our previous hierarchical model attempted to infer the extent of such sharing by estimating the proportion of eQTLs which were shared in various `configurations' or patterns of binary activity in which a QTL was `called' or `absent' in each tissue.
 However, these binary configurations were still \textit{limited in their ability to capture continuous variation} in levels of activity among tissues in which the SNP is considered active. 
 In fact, as the number of tissues considered increases, perhaps the more interesting and biologically relevant question becomes one of quantitative heterogeneity - that is, how do the patterns of effect vary across tissues in which the SNP is called `active'.  

The novel approach we offer here allows groups of QTLs to be classified not only by their presence or absence in a particular tissue, but by their  \textit{relationships in continuous effects between tissues}, e.g., consistently larger effects in some tissues than other. Critically, quantifying the effect sizes of the gene-snp pair across tissues considering the evidence contained in all tissues jointly thus reveals new patterns of activity across tissues, which differ in their relationship in sign and magnitude within and between tissues. The novel framework described here allows that these effects be `shared' but not necessarily `consistent' across tissues, thus capturing continuous variation heretofore missed. The structure of this paper is as follows: we will describe our approach in brief for modeling and estimating these effect sizes across tissues, demonstrate the utility of such an approach on simulated data, and apply our method to data from the GTEX dataset, version 6.0, where we analyze effects size for 16,069 genes across 44 tissues. In so doing, we offer novel insight into the patterns of quantitative heterogeneity in genetic effects among a greater number of tissues than ever analyzed before.
